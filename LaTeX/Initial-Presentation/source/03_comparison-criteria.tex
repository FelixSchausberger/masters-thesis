\begin{frame}{Comparison criteria}%
    \begin{columns}[T,onlytextwidth]%
        \begin{column}[T]{0.48\textwidth}%
            \begin{itemize}
                \item Maximum allowable perturbation~\cite{Rummel2008}~\cite{Maus2010}
                    \begin{itemize}
                        \item Limit cycle: $\upsilon_{n + 1} = \upsilon_n$
                        \item Poincaré map: $\upsilon_{n + 1} = S(\upsilon)$
                    \end{itemize}
                \item Disturbance response % recovery process of the model after a disturbance: how fast the model returns to a limit cycle.
                \item Mechanical cost analysis
            \end{itemize}
            Possible disturbances:
            \begin{itemize}
                \item Horizontal velocity $\dot{x}_0$ % (instantaneous speed increase/decrease of the \acrshort{com})
                \item Floor height $\dot{y}_0$
                \item Torso angular rate $\dot{\phi}_0$
            \end{itemize}
        \end{column}%
        \begin{column}[T]{0.48\textwidth}%
            \begin{figure}[htb]%
                \centering%
                \includestandalone{figures/disturbance-rejection/disturbance-rejection}%
                \caption{Determination of the maximum allowable disturbance rejection. (Adapted from~\cite{Bommel2011}, p. 5)}%
                \label{fig:disturbance-rejection}%
            \end{figure}%
        \end{column}%
    \end{columns}%
\end{frame}%
%
\note[itemize]{%
    \item Disturbance rejection: maximum allowable perturbation of model
    \item First, limit cycle, which model is in when initial states remain unchanged for consecutive strides, will get determined
    \item Important role: Poincaré (Púokaré) map: Mapping between initial states of consecutive strides
    \item Explain model
    \item Other possible comparison criteria: Disturbance response: Recovery process of model after disturbance,  mechanical cost analysis
    \item Possible disturbances: $x_0$, $y_0$, $\phi_0$: Introduced as single error on initial state of limit cycle
}%
%
% (Bommel, 2011): The initial conditions define the set of state parameters for which the running model is in limit cycle, which means the runner is in a nominally periodic, stable sequence of hops. The initial state of a consecutive hop is determined by the end state of the previous hop. The mapping between the initial states of consecutive hops is called a Poincaré map.
%
% The model is in limit cycle when the initial states remain unchanged for consecutive hops
%
% limit cycle is found by minimising the difference between the initial state v0 and the state of the running model after one step v1
%
% Disturbances are introduced to the model as a single error on the initial state of the limit cycle.