% !TeX spellcheck = en_US
\chapter{Introduction}%

Robotics aims to build artificial cognitive systems that can act on their own to achieve some predefined goal. However, in order to be able to respond to and interact with one's environment, it is necessary to grasp the basic mechanisms of our world. To master this, agents\footnote{Respectively any artificial entity displaying some degree of cognition~\cite{Vernon2014}.} are required to perceive their environment, anticipate the need to act, learn from experience, and adapt to changing circumstances. In order to be able to perform tasks imposed by humans, mobile robots must navigate a complex, uncertain, unstructured and human-shaped environment. This can only be achieved by exhibiting some degree of cognition.~\cite{Vernon2014} Hereby, nature provides a remedy. The course of millions of years of evolution created an almost inexhaustible arsenal of potential solutions and highly optimized system processes, which frequently inspire engineers.~\cite{Siciliano2016} \\

Studies of mechatronic systems inspired by biology can be categorized in respects of locomotion and mechanisms, actuation, sensing, and control~\cite{Sitti2013}. Especially nature's ways of mastering locomotion provide a myriad of inspirations as nature has evolved various biological forms and functions to maneuver energy efficiently, agilely and safely through even the most hazardous environments. Locomotion is equally fundamental to all living things for foraging, catching prey, evading predators, protecting territory, finding mates, and migrating. For mankind in particular, it has played a central role in hunting, agriculture, transportation, sports, and warfare.~\cite{Lee2018} As with any structurally or functionally occurring feature in nature, it is important to observe locomotion through the lens of organic evolution, as it emerged through the process of natural selection rather than through the mostly straightforward process of engineering. Subsequently, two main types of terrestrial legged locomotion have established in nature: walking upright on at least two legs like humans and most mammals do, which facilitates fast locomotion; and crawling low over the ground like reptiles, which usually tends to greater stability especially on rough terrain.~\cite{Sitti2013} Hereby, energy comes at a premium not only for living creatures but also for robots, which need to carry sufficient energy in their batteries. Although the energy of a robot is consumed at many levels, from the control systems to the actuators,~\citeauthor{Lee2018}~\cite{Lee2018} assume that the~\glsxtrfull{mcot} is an integral energy cost. Measuring this consumption allows the most direct comparison between the gaits of legged creatures and robots. Although legged robots have equaled or even surpassed the total~\glsxtrfull{cot} of legged creatures, this is usually only achieved by choosing extremely slow speeds or by using regenerative mechanisms.~\cite{Lee2018}\\

The goal of this work is to stabilize the torso of a bipedal robot and thus enable free movement of the upper body on a plane, as the trunk was previously fixed with rather stiff springs and hence unable to swing forward or backward with previous walking patterns. In the course of this, correlations of parameters that influence the periodic gait shall be found. By simulating and analyzing the dynamics of the entire system, an attempt is made to deduce a control law. In order to be able to select a suitable method, it is necessary to understand which gait types have established in nature, their characteristics and how they can be distinguished. Chapter~\ref{sec:fundamentals} gives an overview of this and then presents different possibilities, from which the~\glsxtrfull{mca} has emerged as the most promising approach and is thus pursued further. Subsequently, a biologically inspired controller is implemented, with the help of which a periodic gait can be achieved. The robot, controller, and further methods used are presented in chapter~\ref{sec:implementation}. Finally, quantitative parameters based on the~\glsxtrshort{mca}, which include the previously mentioned~\glsxtrshort{mcot}, are determined to evaluate the controller.  The results are presented in chapter~\ref{sec:evaluation}. Since a periodic although not energy efficient gait was found, an outlook with possible improvements is given in the final chapter~\ref{sec:conclusion}.
