% In total max. 1 Page!
\AMstudentthesisAbstract{%
%
Balancing the trunk on long legs is a major challenge in bipedal walking. To avoid conflicting control objectives, all aspects of gait control must be tightly integrated. The goal of this thesis is to quantify correlations of parameters that influence the periodic gait. In the course of this, similarities and differences between different control strategies will be analyzed and tested in a bipedal robot simulation. At first, a review of approaches for torso stabilization in bipedal robots will be given. A selected controller will then be implemented and simulated in the given computer model of the JenaFox bipedal walker. Finally, quantitative parameters based on the~\glsxtrlong{mca} are determined to evaluate the controller. For this purpose, trajectory optimization is used, to adjust the initial system dynamics in a way that the robot achieves a periodic gait. The designed controller has strong influence of a neural network controller. The concept of~\glsxtrlong{vlo} is used to divide the periodic gait into a phase that repetitively results in a meaningful gait. Finally, a stability analysis is performed, which shows that the gait is asymptotically stable but not energy efficient. Altogether, this thesis investigates control strategies for balancing the trunk in bipedal walking and provides a solution approach that ensures the most periodic gait possible.
%
}{%
%
Das Ausbalancieren des Rumpfes auf langen Beinen stellt eine große Herausforderung beim zweibeinigen Gehen dar. Um widersprüchliche Steuerungsziele zu vermeiden, müssen alle Aspekte der Gangsteuerung eng miteinander verknüpft werden. Das Ziel dieser Arbeit ist es, Korrelationen von Parametern zu quantifizieren, die den periodischen Gang beeinflussen. Im Zuge dessen werden Gemeinsamkeiten und Unterschiede zwischen verschiedenen Kontrollstrategien analysiert und in einer zweibeinigen Robotersimulation getestet. Zunächst wird ein Überblick über Ansätze zur Torsostabilisierung bei zweibeinigen Robotern gegeben. Ein ausgewählter Regler wird dann in dem gegebenen Computermodell des JenaFox Zweibeiners implementiert und simuliert. Abschließend werden quantitative Parameter auf Basis der~\glsxtrlong{mca} ermittelt, um den Regler zu bewerten. Zu diesem Zweck wird eine Trajektorienoptimierung eingesetzt, um die anfängliche Systemdynamik so einzustellen, dass der Roboter einen periodischen Gang erreicht. Der entworfene Regler hat einen starken Einfluss eines neuronalen Netzcontrollers. Das Konzept der~\glsxtrlong{vlo} wird verwendet, um den periodischen Gang in eine Phase zu unterteilen, die wiederholt zu einem sinnvollen Gang führt. Abschließend wird eine Stabilitätsanalyse durchgeführt, die zeigt, dass der Gang zwar asymptotisch stabil, aber nicht energieeffizient ist. Insgesamt werden in dieser Arbeit Regelungsstrategien für das Ausbalancieren des Rumpfes beim zweibeinigen Gehen untersucht und ein Lösungsansatz geliefert, der einen möglichst periodischen Gang gewährleistet.
%
}%
%
%
