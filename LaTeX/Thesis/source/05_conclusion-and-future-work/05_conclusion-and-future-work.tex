% !TeX spellcheck = en_US
\chapter{Conclusion and future work}%
\label{sec:conclusion}

By simulating and analyzing the dynamics of the entire system a control law was deduced, that sets the~\glsxtrshort{aea} in relation to~\gls{rho} in a way that the periodicity of the gait is preserved in the subsequent step. Over trajectory optimization, choosing parameters according to appropriate experimental data and sweeping within a reasonable range, a set of parameters, which lead to an asymptotically stable gait, could have been found. \\

The presented optimizer used to tune the gait could even be the starting point to develop new promising gaits. A first step could be to generate a basin of attraction to analyze the error distance of the solution to the most periodic solution. Another future improvement could be an iterative optimization by taking the gait learned here, which is valid but not energy efficient, as the basis for another optimization procedure and to further refine the desired gait with additional penalty functions and constraints. Furthermore, the trajectory optimization could be extended by a Hermit-Simpson collocation, which is a numerical method for solving~\glsxtrfullpl{ode} by approximating the solution using piecewise polynomials, based on two well-known numerical methods: the Hermite interpolation and the Simpson's rule for integration. The method is particularly useful for solving stiff~\glsxtrshortpl{ode} with widely varying time scales, as it can handle rapid changes in the solution without requiring a high temporal resolution.~\cite{Kelly2017}  \\

An ongoing issue that should be addressed in further work is the extreme oscillations of the horizontal velocity (see figure~\ref{fig:displacement-full}), which also include a negative horizontal displacement of the robot (see figure~\ref{fig:displacement-zoom}). This results in an enormous energy demand, as the robot permanently accelerates and then decelerates again with the given gait. A possible solution approach would be to give the optimizer stricter constraints, for example, that the robot is not allowed to reach negative horizontal velocities at all ($\gls{v}_{\glsxtrshort{com}_x} > 0\,\textrm{\si{\metre\per\second}}~\forall~\gls{t}$). However, at the current time of writing, no reasonable gait could have been found that takes stricter constraints into account. Finally, further objective functions can be tried, to ensure good convergence of the~\glsxtrshort{nlp}. Among others, the~\glsxtrshort{cot} is a widely used objective function, which, however, is difficult to optimize because the solutions tend to have discontinuities.~\cite{Kelly2017}
